\documentclass{book}
\usepackage{fontspec}
\setmainfont{STIX Two Text}

%PACKAGES
\iffalse
Here are the packages that I use
\fi

\usepackage{blindtext, hyperref, verbatim, minted, graphicx, amssymb, textcomp, enumerate, tcolorbox, newunicodechar, textgreek, wasysym, tipa, eso-pic, lipsum, bbold, dsfont}
\usepackage[margin=1.3in]{geometry}
\usepackage{longtable}
\usepackage{newunicodechar}
\usepackage{amsthm}
\usepackage{tikz}
\usepackage{tikz-cd}






%ENVIRONMENTS

%Here I define some common environments. I use definitions, theorems, examples, and lemmas.


\theoremstyle{definition}
\newtheorem{definition}{Definition}
\newtheorem{theorem}{Theorem}
\newtheorem{example}{Example}
\newtheorem{lemma}{Lemma}


\newunicodechar{ₙ}{${}_{n}$}

\newunicodechar{𝓓}{$\mathcal{D}$}
\newunicodechar{∂}{$\partial$}

%\newunicodechar{π⃗}{$\stackrel{\arr}{\pi}$}

\newunicodechar{×}{$\times$}
\newunicodechar{→}{$\rightarrow$}
\newunicodechar{⟨}{$\langle$}
\newunicodechar{⟩}{$\rangle$}
\newunicodechar{↦}{$\mapsto$}
\newunicodechar{∧}{$\wedge$}
\newunicodechar{∨}{$\vee$}
\newunicodechar{∃}{$\exists$}
\newunicodechar{∀}{$\forall$}
\newunicodechar{¬}{$\neg$}
\newunicodechar{ᵃ}{${}^{\texttt{a}}$}
\newunicodechar{ᵇ}{${}^{\texttt{b}}$}
\newunicodechar{ᶜ}{${}^{\texttt{c}}$}
\newunicodechar{ᵈ}{${}^{\texttt{d}}$}
\newunicodechar{ᵉ}{${}^{\texttt{e}}$}
\newunicodechar{ᶠ}{${}^{\texttt{f}}$}
\newunicodechar{ᵍ}{${}^{\texttt{g}}$}
\newunicodechar{ʰ}{${}^{\texttt{h}}$}
\newunicodechar{ⁱ}{${}^{\texttt{i}}$}
\newunicodechar{ʲ}{${}^{\texttt{j}}$}
\newunicodechar{ᵏ}{${}^{\texttt{k}}$}
\newunicodechar{ˡ}{${}^{\texttt{l}}$}
\newunicodechar{ᵐ}{${}^{\texttt{m}}$}
\newunicodechar{ⁿ}{${}^{\texttt{n}}$}
\newunicodechar{ᵒ}{${}^{\texttt{o}}$}
\newunicodechar{ᵖ}{${}^{\texttt{ω}}$}
\newunicodechar{ʳ}{${}^{\texttt{r}}$}
\newunicodechar{ˢ}{${}^{\texttt{s}}$}
\newunicodechar{ᵗ}{${}^{\texttt{t}}$}
\newunicodechar{ᵘ}{${}^{\texttt{u}}$}
\newunicodechar{ᵛ}{${}^{\texttt{v}}$}
\newunicodechar{ʷ}{${}^{\texttt{w}}$}
\newunicodechar{ˣ}{${}^{\texttt{x}}$}
\newunicodechar{ʸ}{${}^{\texttt{y}}$}
\newunicodechar{ᶻ}{${}^{\texttt{z}}$}
\newunicodechar{⁰}{${}^{\texttt{0}}$}
\newunicodechar{¹}{${}^{\texttt{1}}$}
\newunicodechar{²}{${}^{\texttt{2}}$}
\newunicodechar{³}{${}^{\texttt{3}}$}
\newunicodechar{⁴}{${}^{\texttt{4}}$}
\newunicodechar{⁵}{${}^{\texttt{5}}$}
\newunicodechar{⁶}{${}^{\texttt{6}}$}
\newunicodechar{⁷}{${}^{\texttt{7}}$}
\newunicodechar{⁸}{${}^{\texttt{8}}$}
\newunicodechar{⁹}{${}^{\texttt{9}}$}
\newunicodechar{⁻}{${}^{\texttt{-}}$}
\newunicodechar{ᵒ}{${}^{\texttt{o}}$}
\newunicodechar{ᵖ}{${}^{\texttt{ω}}$}
\newunicodechar{⁻}{${}^{\texttt{-}}$}
\newunicodechar{¹}{${}^{\texttt{1}}$}
\newunicodechar{₀}{${}_{\texttt{0}}$}
\newunicodechar{₁}{${}_{\texttt{1}}$}
\newunicodechar{₂}{${}_{\texttt{2}}$}
\newunicodechar{₃}{${}_{\texttt{3}}$}
\newunicodechar{₄}{${}_{\texttt{4}}$}
\newunicodechar{₅}{${}_{\texttt{5}}$}
\newunicodechar{₆}{${}_{\texttt{6}}$}
\newunicodechar{₇}{${}_{\texttt{7}}$}
\newunicodechar{₈}{${}_{\texttt{8}}$}
\newunicodechar{₉}{${}_{\texttt{9}}$}
\newunicodechar{𝔸}{$\mathbb{A}$}
\newunicodechar{𝔹}{$\mathbb{B}$}
\newunicodechar{ℂ}{$\mathbb{C}$}
\newunicodechar{𝔻}{$\mathbb{D}$}
\newunicodechar{𝔼}{$\mathbb{E}$}
\newunicodechar{𝔽}{$\mathbb{F}$}
\newunicodechar{𝔾}{$\mathbb{G}$}
\newunicodechar{ℍ}{$\mathbb{H}$}
\newunicodechar{𝕀}{$\mathbb{I}$}
\newunicodechar{𝕁}{$\mathbb{J}$}
\newunicodechar{𝕂}{$\mathbb{K}$}
\newunicodechar{𝕃}{$\mathbb{L}$}
\newunicodechar{𝕄}{$\mathbb{M}$}
\newunicodechar{ℕ}{$\mathbb{N}$} 
\newunicodechar{𝕆}{$\mathbb{O}$}
\newunicodechar{ℙ}{$\mathbb{P}$}
\newunicodechar{ℚ}{$\mathbb{Q}$}
\newunicodechar{ℝ}{$\mathbb{R}$}
\newunicodechar{𝕊}{$\mathbb{S}$}
\newunicodechar{𝕋}{$\mathbb{T}$} 
\newunicodechar{𝕌}{$\mathbb{U}$}
\newunicodechar{𝕍}{$\mathbb{V}$}
\newunicodechar{𝕎}{$\mathbb{W}$}
\newunicodechar{𝕏}{$\mathbb{X}$}
\newunicodechar{𝕐}{$\mathbb{Y}$}
\newunicodechar{ℤ}{$\mathbb{Z}$}
\newunicodechar{𝕒}{$\mathbb{a}$}
\newunicodechar{𝕓}{$\mathbb{b}$}
\newunicodechar{𝕔}{$\mathbb{c}$}
\newunicodechar{𝕕}{$\mathbb{d}$}
\newunicodechar{𝕖}{$\mathbb{e}$}
\newunicodechar{𝕗}{$\mathbb{f}$}
\newunicodechar{𝕘}{$\mathbb{g}$}
\newunicodechar{𝕙}{$\mathbb{h}$}
\newunicodechar{𝕚}{$\mathbb{i}$}
\newunicodechar{𝕛}{$\mathbb{j}$}
\newunicodechar{𝕜}{$\mathbb{k}$}%𝔸𝔹ℂ𝔻𝔼𝔽𝔾ℍ𝕀𝕁𝕂𝕃𝕄ℕ𝕆ℙℚℝ𝕊𝕋𝕌𝕍𝕎𝕏𝕐ℤ𝕒𝕓𝕔𝕕𝕖𝕗𝕘𝕙𝕚𝕛𝕜𝕝𝕞𝕟𝕠𝕡𝕢𝕣𝕤𝕥𝕦𝕧𝕨𝕩𝕪𝕫
\newunicodechar{𝕝}{$\mathbb{l}$} 
\newunicodechar{𝕞}{$\mathbb{m}$}
\newunicodechar{𝕟}{$\mathbb{n}$}
\newunicodechar{𝕠}{$\mathbb{o}$}
\newunicodechar{𝕡}{$\mathbb{p}$}
\newunicodechar{𝕢}{$\mathbb{q}$}
\newunicodechar{𝕣}{$\mathbb{r}$}
\newunicodechar{𝕤}{$\mathbb{s}$}
\newunicodechar{𝕥}{$\mathbb{t}$}
\newunicodechar{𝕦}{$\mathbb{u}$}
\newunicodechar{𝕧}{$\mathbb{v}$}
\newunicodechar{𝕨}{$\mathbb{w}$}
\newunicodechar{𝕩}{$\mathbb{x}$}
\newunicodechar{𝕪}{$\mathbb{y}$}
\newunicodechar{𝕫}{$\mathbb{z}$}
\newunicodechar{𝚫}{$\Delta$}
\newunicodechar{ʃ}{$\int$}
\newunicodechar{∪}{$\cup$}
\newunicodechar{∩}{$\cap$}
\newunicodechar{±}{$\pm$}
\newunicodechar{𝔄}{$\mathfrak{A}$}




\newunicodechar{𝔅}{$\mathfrak{B}$}
\newunicodechar{ℭ}{$\mathfrak{C}$}
\newunicodechar{𝔇}{$\mathfrak{D}$}
\newunicodechar{𝔈}{$\mathfrak{E}$}
\newunicodechar{𝔉}{$\mathfrak{F}$}
\newunicodechar{𝔊}{$\mathfrak{G}$}
\newunicodechar{ℌ}{$\mathfrak{H}$}
\newunicodechar{ℑ}{$\mathfrak{I}$}
\newunicodechar{𝔍}{$\mathfrak{J}$}
\newunicodechar{𝔎}{$\mathfrak{K}$}
\newunicodechar{𝔏}{$\mathfrak{L}$}
\newunicodechar{𝔐}{$\mathfrak{M}$}
\newunicodechar{𝔑}{$\mathfrak{N}$}
\newunicodechar{𝔒}{$\mathfrak{O}$}
\newunicodechar{𝔓}{$\mathfrak{P}$}
\newunicodechar{𝔔}{$\mathfrak{Q}$}
\newunicodechar{ℜ}{$\mathfrak{R}$}
\newunicodechar{𝔖}{$\mathfrak{S}$}
\newunicodechar{𝔗}{$\mathfrak{T}$}
\newunicodechar{𝔘}{$\mathfrak{U}$}
\newunicodechar{𝔙}{$\mathfrak{V}$}
\newunicodechar{𝔚}{$\mathfrak{W}$}
\newunicodechar{𝔛}{$\mathfrak{X}$}
\newunicodechar{𝔜}{$\mathfrak{Y}$}
\newunicodechar{ℨ}{$\mathfrak{Z}$}

\newunicodechar{𝔞}{$\mathfrak{a}$}
\newunicodechar{𝔟}{$\mathfrak b$}
\newunicodechar{𝔠}{$\mathfrak{c}$}
\newunicodechar{𝔡}{$\mathfrak{d}$}
\newunicodechar{𝔢}{$\mathfrak{e}$}
\newunicodechar{𝔣}{$\mathfrak{f}$}
\newunicodechar{𝔤}{$\mathfrak{g}$}
\newunicodechar{𝔥}{$\mathfrak{h}$}
\newunicodechar{𝔦}{$\mathfrak{i}$}
\newunicodechar{𝔧}{$\mathfrak{j}$}
\newunicodechar{𝔨}{$\mathfrak{k}$}
\newunicodechar{𝔩}{$\mathfrak{l}$}
\newunicodechar{𝔪}{$\mathfrak{m}$}
\newunicodechar{𝔫}{$\mathfrak{n}$}
\newunicodechar{𝔬}{$\mathfrak{o}$}
\newunicodechar{𝔭}{$\mathfrak{ω}$}
\newunicodechar{𝔮}{$\mathfrak{q}$}
\newunicodechar{𝔯}{$\mathfrak{r}$}
\newunicodechar{𝔰}{$\mathfrak{s}$}
\newunicodechar{𝔱}{$\mathfrak{t}$}
\newunicodechar{𝔲}{$\mathfrak{u}$}
\newunicodechar{𝔳}{$\mathfrak{v}$}
\newunicodechar{𝔴}{$\mathfrak{w}$}
\newunicodechar{𝔵}{$\mathfrak{x}$}
\newunicodechar{𝔶}{$\mathfrak{y}$}
\newunicodechar{𝔷}{$\mathfrak{z}$}

\newunicodechar{𝐀}{${\bf{A}}$}
\newunicodechar{𝐁}{${\bf{B}}$}
\newunicodechar{𝐂}{${\bf{C}}$}
\newunicodechar{𝐃}{${\bf{D}}$}
\newunicodechar{𝐄}{${\bf{E}}$}
\newunicodechar{𝐅}{${\bf{F}}$}
\newunicodechar{𝐆}{${\bf{G}}$}
\newunicodechar{𝐇}{${\bf{H}}$}
\newunicodechar{𝐈}{${\bf{I}}$}
\newunicodechar{𝐉}{${\bf{J}}$}
\newunicodechar{𝐊}{${\bf{K}}$}
\newunicodechar{𝐋}{${\bf{L}}$}
\newunicodechar{𝐌}{${\bf{M}}$}
\newunicodechar{𝐍}{${\bf{N}}$}
\newunicodechar{𝐎}{${\bf{O}}$}
\newunicodechar{𝐏}{${\bf{P}}$}
\newunicodechar{𝐐}{${\bf{Q}}$}
\newunicodechar{𝐑}{${\bf{R}}$}
\newunicodechar{𝐒}{${\bf{S}}$}
\newunicodechar{𝐓}{${\bf{T}}$}
\newunicodechar{𝐔}{${\bf{U}}$}
\newunicodechar{𝐕}{${\bf{V}}$}
\newunicodechar{𝐖}{${\bf{W}}$}
\newunicodechar{𝐗}{${\bf{X}}$}
\newunicodechar{𝐘}{${\bf{Y}}$}
\newunicodechar{𝐙}{${\bf{Z}}$}

\newunicodechar{𝐚}{${\bf{a}}$}
\newunicodechar{𝐛}{${\bf{b}}$}
\newunicodechar{𝐜}{${\bf{c}}$}
\newunicodechar{𝐝}{${\bf{d}}$}
\newunicodechar{𝐞}{${\bf{e}}$}
\newunicodechar{𝐟}{${\bf{f}}$}
\newunicodechar{𝐠}{${\bf{g}}$}
\newunicodechar{𝐡}{${\bf{h}}$}
\newunicodechar{𝐢}{${\bf{i}}$}
\newunicodechar{𝐣}{${\bf{j}}$}
\newunicodechar{𝐤}{${\bf{k}}$}
\newunicodechar{𝐥}{${\bf{l}}$}
\newunicodechar{𝐦}{${\bf{m}}$}
\newunicodechar{𝐧}{${\bf{n}}$}
\newunicodechar{𝐨}{${\bf{o}}$}
\newunicodechar{𝐩}{${\bf{ω}}$}
\newunicodechar{𝐪}{${\bf{q}}$}
\newunicodechar{𝐫}{${\bf{r}}$}
\newunicodechar{𝐬}{${\bf{s}}$}
\newunicodechar{𝐭}{${\bf{t}}$}
\newunicodechar{𝐮}{${\bf{u}}$}
\newunicodechar{𝐯}{${\bf{v}}$}
\newunicodechar{𝐰}{${\bf{w}}$}
\newunicodechar{𝐱}{${\bf{x}}$}
\newunicodechar{𝐲}{${\bf{y}}$}
\newunicodechar{𝐳}{${\bf{z}}$}

\newunicodechar{⊣}{\ensuremath{\dashv}}
\newunicodechar{ॱ}{${}^{\cdot}$}
\newunicodechar{𛲔}{${}_{\cdot}$}
\newunicodechar{⋯}{$\cdots$}
\newunicodechar{⇄}{$\rightleftarrows$}
\newunicodechar{⇆}{$\leftrightarrows$}

\newunicodechar{ꜝ}{$\raisebox{1ex}{\scalebox{0.5}{\texttt{!}}}$}
\newunicodechar{ꜞ}{$\raisebox{1ex}{\scalebox{0.5}{\texttt{¡}}}$}



%This is notation we will use for categories


\newunicodechar{𝟙}{$\mathbb{1}$}
\newunicodechar{∘}{$\circ$}

%This is notation we will use for twocategories


\newunicodechar{𝟏}{${\bold{1}}$}
\newunicodechar{⭢}{$\longrightarrow$}
\newunicodechar{•}{${\bullet}$}
\newunicodechar{∙}{${\bullet}$}

%This is notation we will use for ∞-ℂ𝕒𝕥

\newunicodechar{よ}{$\includegraphics[width=0.27cm,height=0.27cm]{yon.png}$}
\newunicodechar{⊥}{$\bot$}
\newunicodechar{∼}{$\sim$}
\newunicodechar{≃}{$\simeq$}
\newunicodechar{≅}{$\cong$}
\newunicodechar{∞}{$\infty$}

\newunicodechar{α}{$\alpha$}
\newunicodechar{β}{$\beta$}
\newunicodechar{γ}{$\gamma$}
\newunicodechar{δ}{$\delta$}
\newunicodechar{ε}{$\epsilon$}
\newunicodechar{η}{$\eta$}
\newunicodechar{ζ}{$\zeta$}
\newunicodechar{θ}{$\theta$}
\newunicodechar{ι}{$\iota$}
\newunicodechar{μ}{$\mu$}
\newunicodechar{κ}{$\kappa$}
\newunicodechar{λ}{$\lambda$}
\newunicodechar{ρ}{$\rho$}
\newunicodechar{π}{$\pi$}
\newunicodechar{σ}{$\sigma$}
\newunicodechar{τ}{$\tau$}
\newunicodechar{υ}{$\upsilon$}
\newunicodechar{φ}{$\phi$}
\newunicodechar{ψ}{$\psi$}
\newunicodechar{ξ}{$\xi$}
\newunicodechar{χ}{$\chi$}
\newunicodechar{ω}{$\omega$}

\newunicodechar{⊗}{$\otimes$}

\makeatletter
\newcommand*{\shifttext}[2]{\settowidth{\@tempdima}{#2}\makebox[\@tempdima]{\hspace*{#1}#2}}
\makeatother
\definecolor{Red}{cmyk}{0.1, 0.70, 0.65, 0.00, 1.00}
\definecolor{Blue}{cmyk}{0.9, 0.2, 0.2, 0.00, 1.00}
\definecolor{Yellow}{cmyk}{0.0, 0.00, 0.7, 0.00, 0.5}
\definecolor{Green}{cmyk}{0.6, 0.0, 0.6, 0.00, 1.00}
\definecolor{Purple}{cmyk}{0.8, 0.3, 0.3, 0.00, 1.00}
\definecolor{Orange}{cmyk}{0.0, 0.3, 0.7, 0.00, 1.00}
\definecolor{Grey}{cmyk}{0.13, 0.13, 0.13, 0.00, 1.00}
\newcounter{definitioncounter}
\setcounter{definitioncounter}{1}
\newcounter{theoremcounter}
\setcounter{theoremcounter}{1}
\newcounter{printcounter}
\setcounter{printcounter}{1}
\newcounter{examplecounter}
\setcounter{examplecounter}{1}
\newcounter{ccounter}
\setcounter{ccounter}{1}
\newcounter{pcounter}
\setcounter{pcounter}{1}
\newcounter{lcounter}
\setcounter{lcounter}{1}
\newcounter{sectioncount}
\newcounter{subsectioncount}
\setcounter{sectioncount}{1}
\renewcommand{\section}[1]{\newpage\ \\ \ \\ \begin{center} \scalebox{1.5}{\texttt{\thesectioncount . #1}} \stepcounter{sectioncount} \setcounter{subsectioncount}{1} \end{center} \begin{center} \ \\ \ \\ \thispagestyle{empty} \end{center}}
\renewcommand{\subsection}[1]{\texttt{\thesubsectioncount . #1} \stepcounter{subsectioncount}}
\renewcommand{\backslash}{\reflectbox{\texttt{/}}}

\newcounter{chaptercount}
\renewcommand{\chapter}[1]{
\newpage
{
\Huge 
\begin{center}
\ \\
\ \\
\thispagestyle{empty}
\texttt{#1}
\end{center}}
\ \\
\ \\
}

\newcounter{partcount}
\stepcounter{partcount}
\renewcommand{\part}[1]{
\newpage
{
\Huge 
\begin{center}
\ \\
\ \\
\ \\
\ \\
\ \\
\ \\
\thispagestyle{empty}
\texttt{PART {\thepartcount}: #1}
\stepcounter{partcount}
\end{center}}
\ \\
\ \\
}
\begin{document}

\thispagestyle{empty} 

\AddToShipoutPicture*
    {\put(540,720){

    \href{http://www.linearlibrary.net}{\includegraphics[width=2cm,height=2cm]{ll.png}}

    }}

\AddToShipoutPicture*
  {\put(470,767){
    \href{https://github.com/linlib/CategoriesandHilbertSpaces/StringDiagramGenerator.py}{\texttt{.py file}}
  }}

\AddToShipoutPicture*
  {\put(470,752){
    \href{https://github.com/linlib/ThreeWhiteheadTheoremsandThreePuppeSequences/ThreeWhiteheadTheoremsandThreePuppeSequences.tex}{\texttt{.tex file}}\\

  }}


\AddToShipoutPicture*
  {\put(470,737){

    \href{http://linearlibrary.net/ThreeWhiteheadTheoremsandThreePuppeSequences/ThreeWhiteheadTheoremsandThreePuppeSequences.pdf}{\texttt{.pdf file}}\\

  }}

  \AddToShipoutPicture*
  {\put(470,722){
    \href{https://github.com/linlib/ThreeWhiteheadTheoremsandThreePuppeSequences/ThreeWhiteheadTheoremsandThreePuppeSequences.lean}{\texttt{.lean file}}

  }}

\ \\

%LEAN: 
\begin{center}
\begin{tcolorbox}[width=6.4in,colback={white},coltitle=white]
\begin{center}
\ \\
\scalebox{3}{Two Theorems in Classifying Space Theory and Two Variations Each}\\
\ \\
\end{center}
\end{tcolorbox}
\end{center}
\ \\

{\footnotesize
\begin{center}
\scalebox{1.1}{
\begin{tabular}{|| l | l ||} 
\hline
\hline
E⃗ : Functor (OperadicCategory ∞-Cat) ∞-Cat & e⃗ : \{C : ∞-Cat \} → \{ D : ∞-Cat \} → (F : ∞-Cat.hom C D) → Functor (OperadicPresheaf (O⃗.obj D)) (∞-Cat⁄D)\\
\hline
B⃗ : Functor (OperadicCategory ∞-Cat) ∞-Cat & b⃗ : \{ C : ∞-Cat \} → \{ D : ∞-Cat \} → (F : ∞-Cat.hom C D) → Functor (OperadicPresheaf (O⃗.obj D)) (∞-Cat⁄D) \\
 \hline
∂⃗ : (C : OperadicCategory ∞-Cat) → ∞-Cat.hom (E⃗.obj C) (B⃗.obj C) & ∇⃗ : \{ C : ∞-Cat \} → \{ D : ∞-Cat \} → (F : ∞-Cat.hom C D) → (∞-Cat⁄D).hom (e⃗.obj F) (b⃗.obj F)\\
\hline
\hline
E⃡ : Functor (OperadicGroupoid ∞-Grpd) ∞-Grpd & e⃡ : \{ X : ∞-Cat \} → \{ Y : ∞-Cat \} → (F : ∞-Cat.hom X Y) → Functor (OperadicGroupoidAction (O⃡.obj Y)) (∞-Grpd⁄Y) \\
\hline
B⃡ : Functor (OperadicGroupoid ∞-Grpd) ∞-Grpd & b⃡ : \{ X : ∞-Grpd \} → \{ Y : ∞-Grpd \} → (F : ∞-Cat.hom X Y) → Functor (OperadicGroupoidAction (O⃡.obj Y)) (∞-Grpd⁄Y) \\
 \hline
∂⃡ : (G : OperadicCategory ∞-Grpd) → ∞-Grpd.hom (E⃡.obj G) (B⃡.obj G) & ∇⃡ : \{ X : ∞-Grpd \} → \{ Y : ∞-Grpd \} → (F : ∞-Grpd.hom X Y) → (∞-Cat⁄D).hom (e⃡.obj F) (b⃡.obj F) \\
 \hline
 \hline 
E : OperadicGroup ∞-Grpd₋₁ ⭢ ∞-Grpd₋₁ & e : \{ X₋₁ : ∞-Grpd₋₁ \} → \{ Y₋₁ : ∞-Grpd₋₁ \} → (F : ∞-Grpd₋₁.hom X₋₁ Y₋₁) → Functor (∞-Grpd₋₁⁄Y₋₁) (OperadicGroupAction (O.obj Y₋₁)) \\
\hline
B : OperadicGroup ∞-Grpd₋₁ ⭢ ∞-Grpd₋₁ & b : \{ X₋₁ : ∞-Grpd₋₁ \} → \{ Y₋₁ : ∞-Grpd₋₁ \} → (F : ∞-Grpd₋₁.hom X₋₁ Y₋₁) → Functor (∞-Grpd₋₁⁄Y₋₁) (OperadicGroupAction (O.obj Y₋₁))\\
\hline
∂ : (G₋₁ : OperadicGroup ∞-Grpd₋₁) → ∞-Grpd₋₁.hom (E.obj G₋₁) (B.obj G₋₁) & ∇ : \{ G₋₁ : ∞-Grpd₀ \} → \{ Y₀ : ∞-Grpd₀ \} → (F : ∞-Grpd₋₁.hom X₋₁ Y₋₁) → (∞-Cat⁄D).hom (e.obj F) (b.obj F)  \\
\hline
\end{tabular}}
\end{center}}

%LEAN: 
\begin{center}
\begin{tcolorbox}[width=4.13in,colback={white},coltitle=white]
\scalebox{1.5}{E. Dean Young}
\end{tcolorbox}
\end{center}

\newpage
\ \\


\section{Contents}



{
\footnotesize
\begin{longtable}{|| l || l ||} 
\hline
\multicolumn{1}{||c||}{$\texttt{Section}$} & \multicolumn{1}{|c||}{$\texttt{Description}$} \\
\hline
\hline
Unfinished & \\
\hline
Contents & \\
\hline
Unicode & \\
\hline
Introduction & \\
\hline \hline
\multicolumn{2}{||c||}{\texttt{PART I: } BASED CONNECTED ∞-GROUPOIDS} \\
\hline \hline
\multicolumn{2}{||c||}{\texttt{Chapter 1: } Operadic Groups and Operadic Group Actions} \\
\hline \hline
OperadicGroups & \\
\hline
OperadicGroupActions & \\
\hline \hline
\multicolumn{2}{||c||}{\texttt{Chapter 2: }The Classifying Space and the Total Space} \\
\hline \hline
E : OperadicGroup ∞-Grpd₋₁ ⭢ ∞-Grpd₋₁ &  \\
\hline
e : \{ X₋₁ : ∞-Grpd₋₁ \} → \{ Y₋₁ : ∞-Grpd₋₁ \} → (F : ∞-Grpd₋₁.hom X₋₁ Y₋₁) → Functor (∞-Grpd₋₁⁄Y₋₁) (OperadicGroupAction (O.obj Y₋₁))  & \\
\hline
B : OperadicGroup ∞-Grpd₋₁ ⭢ ∞-Grpd₋₁ &  \\
\hline
b : \{ X₋₁ : ∞-Grpd₋₁ \} → \{ Y₋₁ : ∞-Grpd₋₁ \} → (F : ∞-Grpd₋₁.hom X₋₁ Y₋₁) → Functor (∞-Grpd₋₁⁄Y₋₁) (OperadicGroupAction (O.obj Y₋₁)) & \\
\hline
∂ : (G₋₁ : OperadicGroup ∞-Grpd₋₁) → ∞-Grpd₋₁.hom (E.obj G₋₁) (B.obj G₋₁) &  \\
\hline
∇ : \{ G₋₁ : ∞-Grpd₀ \} → \{ Y₀ : ∞-Grpd₀ \} → (F : ∞-Grpd₋₁.hom X₋₁ Y₋₁) → (∞-Cat⁄D).hom (e.obj F) (b.obj F) & \\
\hline
\multicolumn{2}{||c||}{\texttt{Chapter 3: }The Recognition Theorem} \\
\hline \hline
 & \\
\hline
 & \\
\multicolumn{2}{||c||}{\texttt{Chapter 4: }The Classifying Space Theorem} \\
\hline \hline
 & \\
\hline
 & \\
\hline \hline
\multicolumn{2}{||c||}{\texttt{PART II: } ∞-GROUPOIDS} \\
\hline \hline
\multicolumn{2}{||c||}{\texttt{Chapter 5: } Operadic Groupoids and Operadic Groupoid Actions} \\
\hline \hline
OperadicGroupoids & \\
\hline 
OperadicGroupoidActions & \\
\hline \hline
\multicolumn{2}{||c||}{\texttt{Chapter 6: }The Recognition Theorem for ∞-Groupoids} \\
\hline \hline
E⃡ : Functor (OperadicGroupoid ∞-Grpd) ∞-Grpd & \\
\hline
e⃡ : \{ X : ∞-Cat \} → \{ Y : ∞-Cat \} → (F : ∞-Cat.hom X Y) → Functor (OperadicGroupoidAction (O⃡.obj Y)) (∞-Grpd⁄Y) & \\
\hline
B⃡ : Functor (OperadicGroupoid ∞-Grpd) ∞-Grpd & \\
\hline
b⃡ : \{ X : ∞-Grpd \} → \{ Y : ∞-Grpd \} → (F : ∞-Cat.hom X Y) → Functor (OperadicGroupoidAction (O⃡.obj Y)) (∞-Grpd⁄Y) & \\
 \hline
∂⃡ : (G : OperadicCategory ∞-Grpd) → ∞-Grpd.hom (E⃡.obj G) (B⃡.obj G) & \\
\hline
∇⃡ : \{ X : ∞-Grpd \} → \{ Y : ∞-Grpd \} → (F : ∞-Grpd.hom X Y) → (∞-Cat⁄D).hom (e⃡.obj F) (b⃡.obj F) & \\
\hline \hline
\multicolumn{2}{||c||}{\texttt{Chapter 7: }The Recognition Theorem for ∞-Groupoids} \\
\hline \hline
 & \\
\hline
 & \\
\multicolumn{2}{||c||}{\texttt{Chapter 8: }The Classifying Space Theorem for ∞-Groupoids} \\
\hline \hline
 & \\
\hline
 & \\
\hline \hline
\multicolumn{2}{||c||}{\texttt{PART III: } ∞-CATEGORIES} \\
\hline \hline
\multicolumn{2}{||c||}{\texttt{Chapter 9: } Operadic Categories and Operadic Presheaves} \\
\hline \hline
OperadicCategory & \\
\hline
OperadicPresheaves & \\
\hline \hline
\multicolumn{2}{||c||}{\texttt{Chapter 10: }The Recognition Theorem for ∞-Categories} \\
\hline \hline
E⃗ : Functor (OperadicCategory ∞-Cat) ∞-Cat  & \\ 
\hline
e⃗ : \{C : ∞-Cat \} → \{ D : ∞-Cat \} → (F : ∞-Cat.hom C D) → Functor (OperadicPresheaf (O⃗.obj D)) (∞-Cat⁄D) & \\
\hline
B⃗ : Functor (OperadicCategory ∞-Cat) ∞-Cat & \\
\hline
b⃗ : \{ C : ∞-Cat \} → \{ D : ∞-Cat \} → (F : ∞-Cat.hom C D) → Functor (OperadicPresheaf (O⃗.obj D)) (∞-Cat⁄D) & \\
 \hline
∂⃗ : (C : OperadicCategory ∞-Cat) → ∞-Cat.hom (E⃗.obj C) (B⃗.obj C) & \\
\hline
∇⃗ : \{ C : ∞-Cat \} → \{ D : ∞-Cat \} → (F : ∞-Cat.hom C D) → (∞-Cat⁄D).hom (e⃗.obj F) (b⃗.obj F) & \\
\hline \hline
\multicolumn{2}{||c||}{\texttt{Chapter 11: }The Recognition Theorem for ∞-Categories} \\
\hline \hline
 & \\
\hline
 & \\
\hline \hline
\multicolumn{2}{||c||}{\texttt{Chapter 12: }The Classifying Space Theorem for ∞-Categories} \\
\hline \hline
 & \\
\hline
 & \\
\hline \hline
\end{longtable}
}



\chapter{Introduction}

\newpage

In ```TheWhiteheadTheoremandTwoVariations''', I will be defining six "internal" structures based on ```Galois Theories''' by Janelidze and Borceux, as well as six ```operadic''' structures 

\iffalse
{\footnotesize
\begin{center}
\scalebox{1.1}{
\begin{tabular}{|| l | l || l | l || } 
\hline
\hline
\multicolumn{2}{||c||}{Internal} & \multicolumn{2}{||c||}{Operadic} \\
\hline
\multicolumn{2}{||c||}{Unitial} & \multicolumn{2}{||c||}{Actional} & \multicolumn{2}{||c||}{Unitial} & \multicolumn{2}{||c||}{Actional} \\
\hline
\hline
$\texttt{InternalCategory : Cat → Cat}$ & $\texttt{InternalPresheaf : (C : Cat) → (InternalCategory C) → Cat}$ & $\texttt{OperadicCategory\_(-) : ∞-Cat\_(-) → ∞-Cat\_(-)}$ & OperadicPresheaf\_(-) : (X : ∞-Cat\_(-)) → ∞-Cat\_(-)}$ \\
\hline
$\texttt{InternalGroupoid\_(-) : Cat\_(-) → Cat\_(-)}$ & $\texttt{InternalGroupoidAction : (C : Cat) → (InternalGroupoid C) → Cat}$& $\texttt{OperadicGroupoid\_(-) : ∞-Cat\_(-) → ∞-Cat\_(-)}$ & OperadicGroupoidAction\_(-) : (OperadicGroupoid\_() C) → (InternalCategory C) → ∞-Cat\_(-) → ∞-Cat\_(-)}$ \\
\hline
$\texttt{InternalGroup\_(-) : Cat\_(-) → Cat\_(-)}$ & $\texttt{InternalGroupAction : (C : Cat) → (InternalGroup C) → Cat}$ & $\texttt{OperadicGroup\_(-) : ∞-Cat\_(-) → ∞-Cat\_(-)}$ & $\texttt{OperadicGroupAction}$\_$\texttt{(-) : ∞-Cat\_(-) → ∞-Cat\_(-)}$\\
 \hline
\end{tabular}}
\end{center}}
\ \\
{\footnotesize
\begin{center}
\scalebox{1.1}{
\begin{tabular}{|| l || l ||} 
\hline
\hline
Ω⃗ : ∞-Cat ⭢ ∞-Cat & ω⃗ : (C : ∞-Cat) → (D : ∞-Cat) → (F : ∞-Cat.hom C D) → (∞-Cat⁄D) ⭢ ∞-Cat \\
\hline
O⃗ : ∞-Cat ⭢ OperadicCategory ∞-Cat & o⃗ : (C : ∞-Cat) → (D : ∞-Cat) → (F : ∞-Cat.hom C D) → (∞-Cat⁄D) ⭢ OperadicPresheaf (O⃗.obj D)\\
\hline
P⃗ : ∞-Cat ⭢ InternalCategory D(∞-Cat) & p⃗ : (C : ∞-Cat) → (D : ∞-Cat) → (F : ∞-Cat.hom C D) → (∞-Cat⁄D) ⭢ InternalPresheaf (P⃗.obj D) \\
\hline
\hline
Ω⃡ : ∞-Grpd ⭢ ∞-Grpd & ω⃡ : (X : ∞-Grpd) → (Y : ∞-Grpd) → (F : ∞-Cat.hom X Y) → (∞-Grpd⁄Y) ⭢ ∞-Grpd \\
\hline
O⃡ : ∞-Grpd ⭢ OperadicGroupoid ∞-Grpd & o⃡ : (X : ∞-Grpd) → (Y : ∞-Grpd) → (F : ∞-Cat.hom X Y) → (∞-Grpd⁄Y) ⭢ OperadicGroupoidAction (O⃡.obj Y) \\
\hline
P⃡ : ∞-Grpd ⭢ InternalGroupoid D(∞-Grpd) & p⃡ : (X : ∞-Grpd) → (Y : ∞-Grpd) → (F : ∞-Cat.hom C D) → (∞-Grpd⁄Y) ⭢ InternalGroupoidAction (P⃡.obj Y)  \\
\hline
\hline
Ω : ∞-Grpd₋₁ ⭢ ∞-Grpd₋₁  & ω : (X₋₁ : ∞-Grpd₋₁) → (Y₋₁ : ∞-Grpd₋₁) → (F : ∞-Grpd₋₁.hom X₋₁ Y₋₁) → (∞-Grpd₋₁⁄Y₋₁) ⭢ ∞-Grpd₋₁ \\
\hline
O : ∞-Grpd₋₁ ⭢ OperadicGroup ∞-Grpd₋₁ & o : (X₋₁ : ∞-Grpd₋₁) → (Y₋₁ : ∞-Grpd₋₁) → (F : ∞-Grpd₋₁.hom X₋₁ Y₋₁) → (∞-Grpd₋₁⁄Y₋₁) ⭢ OperadicGroupAction (O.obj X) \\
\hline
P : ∞-Grpd₋₁ ⭢ InternalGroup D(∞-Grpd₋₁) & p : (X₋₁ : ∞-Grpd₋₁) → (Y₋₁ : ∞-Grpd₋₁) → (F : ∞-Grpd₋₁.hom X₋₁ Y₋₁) → (∞-Grpd₋₁⁄Y₋₁) ⭢ InternalGroupAction (P.obj X₋₁) \\
\hline
 \hline
\end{tabular}}
\end{center}}
\fi



\iffalse
{\footnotesize
\begin{center}
\scalebox{1.1}{
\begin{tabular}{|| l | l ||} 
\hline
\hline
E⃗ : Functor (OperadicCategory ∞-Cat) ∞-Cat & e⃗ : \{C : ∞-Cat \} → \{ D : ∞-Cat \} → (F : ∞-Cat.hom C D) → Functor (OperadicPresheaf (O⃗.obj D)) (∞-Cat⁄D)\\
\hline
B⃗ : Functor (OperadicCategory ∞-Cat) ∞-Cat & b⃗ : \{ C : ∞-Cat \} → \{ D : ∞-Cat \} → (F : ∞-Cat.hom C D) → Functor (OperadicPresheaf (O⃗.obj D)) (∞-Cat⁄D) \\
 \hline
∂⃗ : (C : OperadicCategory ∞-Cat) → ∞-Cat.hom (E⃗.obj C) (B⃗.obj C) & ∇⃗ : \{ C : ∞-Cat \} → \{ D : ∞-Cat \} → (F : ∞-Cat.hom C D) → (∞-Cat⁄D).hom (e⃗.obj F) (b⃗.obj F)\\
\hline
\hline
E⃡ : Functor (OperadicGroupoid ∞-Grpd) ∞-Grpd & e⃡ : \{ X : ∞-Cat \} → \{ Y : ∞-Cat \} → (F : ∞-Cat.hom X Y) → Functor (OperadicGroupoidAction (O⃡.obj Y)) (∞-Grpd⁄Y) \\
\hline
B⃡ : Functor (OperadicGroupoid ∞-Grpd) ∞-Grpd & b⃡ : \{ X : ∞-Grpd \} → \{ Y : ∞-Grpd \} → (F : ∞-Cat.hom X Y) → Functor (OperadicGroupoidAction (O⃡.obj Y)) (∞-Grpd⁄Y) \\
 \hline
∂⃡ : (G : OperadicCategory ∞-Grpd) → ∞-Grpd.hom (E⃡.obj G) (B⃡.obj G) & ∇⃡ : \{ X : ∞-Grpd \} → \{ Y : ∞-Grpd \} → (F : ∞-Grpd.hom X Y) → (∞-Cat⁄D).hom (e⃡.obj F) (b⃡.obj F) \\
 \hline
 \hline 
E : OperadicGroup ∞-Grpd₋₁ ⭢ ∞-Grpd₋₁ & e : \{ X₋₁ : ∞-Grpd₋₁ \} → \{ Y₋₁ : ∞-Grpd₋₁ \} → (F : ∞-Grpd₋₁.hom X₋₁ Y₋₁) → Functor (∞-Grpd₋₁⁄Y₋₁) (OperadicGroupAction (O.obj Y₋₁)) \\
\hline
B : OperadicGroup ∞-Grpd₋₁ ⭢ ∞-Grpd₋₁ & b : \{ X₋₁ : ∞-Grpd₋₁ \} → \{ Y₋₁ : ∞-Grpd₋₁ \} → (F : ∞-Grpd₋₁.hom X₋₁ Y₋₁) → Functor (∞-Grpd₋₁⁄Y₋₁) (OperadicGroupAction (O.obj Y₋₁))\\
\hline
∂ : (G₋₁ : OperadicGroup ∞-Grpd₋₁) → ∞-Grpd₋₁.hom (E.obj G₋₁) (B.obj G₋₁) & ∇ : \{ G₋₁ : ∞-Grpd₀ \} → \{ Y₀ : ∞-Grpd₀ \} → (F : ∞-Grpd₋₁.hom X₋₁ Y₋₁) → (∞-Cat⁄D).hom (e.obj F) (b.obj F)  \\
\hline
\end{tabular}}
\end{center}}
\fi





 and ```ThePuppeSequenceandTwoVariations'''. In ```InternalUniverses''', I considered straightening and unstraightening and three variations of it, which were each considered before and after the application of D(-). This made for the six diagrams depicted on page ???. In this repository, we consider the classifying space B.\\

Let F : C ⭢ D : ∞-Cat.hom C D be an ∞-functor. Given either the C-infinity presheaf in ∞-Cat/C arising from F : ∞-Cat/D or the C-infinity presheaf in ∞-Cat/D arising from Id${}_{C}$ : ∞-Cat/C, we obtain in both cases an internal presheaf in the corresponding derived category. However, not all internal categories D : InternalCategory D(∞-Cat/C) arise from C-InfinityCategory ∞-Cat/C and not all internal presheaves S : InternalPresheaf D D(∞-Cat/C) arise from C-infinity presheaves over some C-infinity category in ∞-Cat/C.\\

In ```InternalUniverses''', we showed the straightening/unstraightening categorical equivalence and three variations using the six Ω-functors and six E-functors, treating the situations before and after the application of D(-) seperately for a total of six goals.\\

In this section, we consider classifying spaces as well as a perspective about remembering information concerning a right or left adjoint applied to a particular functor or object in the following way: E and Ω and their respective five variations give  ```remembrant''' functors E-infinity and Ω-infinity, which each produce internal presheaves in respective derived categories.\\


\begin{center}
\texttt{Plans to prove three variations of the}\\
\texttt{Whitehead theorem of homotopy groups in}\\
\texttt{Lean 4, with extensive use of Mathlib 4}
\end{center}

\thispagestyle{empty}

\iffalse
contractibility of the E's:

- For a locally contractible group?

https://en.wikipedia.org/wiki/Connection_(vector_bundle)
\fi

\iffalse
Next the B can be thought of as pushing all of the information into ∞-Cat
b can be throught of as pushing information into...

MAKE SURE TO INCLUDE //
\fi

\ \\
{\footnotesize
\begin{center}
\scalebox{1.1}{
\begin{tabular}{|| l | l | l || l | l | l ||} 
\hline
E⃗ : ∞-Cat ⭢ ∞-Cat & B⃗ : ∞-Cat ⭢ ∞-Cat & ∂⃗ : ∞-Cat ⭢ ∞-Cat & e⃗ : (C : ∞-Cat) → (D : ∞-Cat) → Adjunction D([C,∞\_(∞-Cat)]) D([D,∞\_(∞-Cat)]) & b⃗ : (C : ∞-Cat) → (D : ∞-Cat) → D([C,∞\_(∞-Cat)]) ⭢ D([D,∞\_(∞-Cat)])&  \\
\hline
E⃡ : ∞-Grpd ⭢ ∞-Grpd & B⃡ : ∞-Grpd ⭢ ∞-Grpd & ∂⃡ : ∞-Grpd ⭢ ∞-Grpd & e⃗ : (X : ∞-Grpd) → (Y : ∞-Grpd) → Adjunction D([X,∞\_(∞-Grpd)]) D([Y,∞\_(∞-Grpd)]) & b⃡ : (X : ∞-Grpd) → (Y : ∞-Grpd) → D([X,∞\_(∞-Grpd)]) ⭢ D([Y,∞\_(∞-Grpd)])&  \\
 \hline
E : ∞-Grpd₀ ⭢ ∞-Grpd₀ & B : ∞-Grpd₀ ⭢ ∞-Grpd₀  & ∂ : ∞-Grpd₀ ⭢ ∞-Grpd₀ & e : (X : ∞-Grpd₀) → (Y: ∞-Grpd₀) → Adjunction D([X,∞\_(∞-Grpd₀)]) D([Y,∞\_(∞-Grpd₀)]) & b : (X : ∞-Grpd₀) → (Y : ∞-Grpd₀) → D([X,∞\_(∞-Grpd₀)]) ⭢ D([Y,∞\_(∞-Grpd₀)])&  \\
 \hline
\end{tabular}}
\end{center}}
 
\newpage

By the time this repository is seen to in 2025, I will have filled out a certain six operadic structures to do with ∞-categories and ∞-groupoids. Each of these structures will be made to work together with Pow {X : Type} : λ(n : ℕ),X → X. The six operadic structures are endofunctions of one of six mathematical objects, here with an option for 12 based on models A and B. 

\begin{center}
\texttt{B¹ : Functor (pow OperadicGroup 2) (pow OperadicGroup 2)}
\end{center}

\begin{center}
\texttt{Bⁿ : Functor (pow OperadicGroup 2) (pow OperadicGroup 2)}
\end{center}

In this repository I construct six categorical equivalences:

\begin{enumerate}
\item 
\item
\item
\item
\item
\item
\end{enumerate}

\iffalse
Continuous functions f : X ⭢ B.obj G correspond to 
\fi


\iffalse

\chapter{ETCC Signature for linlib2024}


\iffalse
\begin{enumerate}
\item NDNI 
\item NDNII 
\item NDNIII 
\end{enumerate}

\begin{enumerate}
\item The linlib signature is five binary digits (Bool × Bool) × (Bool × Bool) × Bool
\item Order : λ(b : Bool × Bool) ↦ (0,0), (0,1), (1,0), (1,1) to 
\item  : 
\item 
\item The third bit is the lowest r such that the object is an (n,r)-category, (n,r)-functor, (n,r)-natural transformation, or equation between (n,r)-natural transformations. We set r = 0 or 1 throughout the project.
\item The fourth bit is "unitial vs. actional".
\item The fifth bit is "lax vs. strict" and divides up our goals based on whether D(-) has been applied.
\end{enumerate}

The table on the next page describes a certain $2^8$ goals for 2024. The full identification number for the project is given by NDb₀b₁rb₃b₄, where b₀ and b₁ are the ND number, $\texttt{r = 0, 1}$ is the filling number, here confined to 0 or 1, b₃ refers to the lax or strict nature of the type, and b₄ refers to its unitial or actional nature.\\

\fi


\begin{center}
\includegraphics[width=0.24\textwidth]{mastertable.png}  \\
\end{center}
\thispagestyle{empty}

\newpage

\ \\

In ```Internal Universes''' I thought about the six variations of straightening and unstraightening featured in the diagrams below:\\


\begin{center}
\includegraphics[width=0.9\textwidth]{ND01.png} \\
\end{center}

\ \\

\begin{center}
\includegraphics[width=0.9\textwidth]{ND11.png} \\
\end{center}

\ \\

\begin{center}
\includegraphics[width=0.9\textwidth]{ND21.png} \\
\end{center}


\newpage

\ \\

\begin{center}
\includegraphics[width=0.9\textwidth]{ND00.png} \\
\end{center}

\ \\

\begin{center}
\includegraphics[width=0.9\textwidth]{ND10.png} \\
\end{center}

\ \\

\begin{center}
\includegraphics[width=0.9\textwidth]{ND20.png} \\
\end{center}


\newpage


6 goals
6 structures

With these goals I want to create several ```remembrant''' adjunctions:

\begin{center}
Ω${}^{∞}$ : ∞-Cat ⇄ C-InfinityCategory ∞-Cat
\end{center}


\begin{enumerate}
\item γ⃗ γ⃡ γ
\item Σ⃗ Σ⃡ Σ
\item σ⃗ σ⃡ σ
\item Pullback of two homs and a single hom vs. a pushout of two products and a single product
\end{enumerate}


 

Ω${}^{∞}$ 

\begin{center}
o⃗ : (C : ∞-Cat) → ∞-Cat⁄C ⭢ OperadicPresheaf (O⃗.obj C)
\end{center}

defining B






\iffalse
% https://tikzcd.yichuanshen.de/#N4Igdg9gJgpgziAXAbVABwnAlgFyxMJZABgBpiBdUkANwEMAbAVxiRAB12cYAPHf4MgDCgJVxAarilAeEScA+gApJAWiF0cASgoBfEJtLpMufIRRkAjFVqMWbae3lLO3PgJXrAIgRDtu-djwEiAMzkFvTMrIggyAAi4lKyjrz8OMAKyqoaXnogGL5GgaTm1KHWEbb2Cc7JaeqabhVJwFFeFjBQAObwRKAAZgBOEAC2SGQgOBBIptQARjBgUEgBI8XhIHKA48CA9BuAKTtqAHQQUwBWAARCOll9gxPUY8PTs-OIi0VWK+uAjpB7Bydn3iCXQyeN3GiEmljCbHqAjkgEngQDrBLsABaDY4AMTUmm2IGoDDoMwYAAUDH5jCAGDBujhsSAZnMFsQ-gC7qMQQAmF4QiJQ5JyGAI5EDNEY97U2mPRaM-qA9ks5nLSFcRLQvlIlHozQiyVXUHAhYckocRWVFLw1WC9Wai5SpAy25A8EG9bbL5HY5RUUPelawFBWWIGVi+n6t5rT77V3uzRaIA
\begin{tikzcd}
{[Cop,infinitySUB(infinity-Cat)]} \arrow[d, "(χ${}_{\cdot}$).obj C", bend left] \arrow[rrr, "(e⃗.hom F)ॱ", bend left] &  &  & {[Dop,infinitySUB(infinity-Cat)]} \arrow[lll, "(e⃗.hom F)ॱ"] \arrow[d, "(χ${}_{\cdot}$).obj D", bend left]               \\
infinitySUB(infinity-Cat)⁄C \arrow[u, "(χ${}^{\cdot}$).obj C", bend left] \arrow[rrr, "\texttt{(ω⃗.hom F)}ॱ"]                   &  &  & infinitySUB(infinity-Cat)⁄D \arrow[lll, "(ω⃗.hom F)𛲔", bend left] \arrow[u, "(χ${}^{\cdot}$).obj D", bend left]
\end{tikzcd}

\fi

\begin{enumerate}
\item It is possible that the B lifts under slightly different conditions than those under which it is an endomorphism.
\item After use of the ∞-box, whose product is difficult, we can invert certain maps to obtain complexes. For this to work we need both biproducts and minus.
\item Not only must these spaces be based; B necessitates that they be A∞ or E∞ (plus some other thing about grouplike, for me).
\item After this we can consider the "free ???", but the product is a bit difficult.
\item [[ℕ,γ⃗], X]
\end{enumerate}


\begin{center}
\includegraphics[width=0.5\textwidth]{littlesquaresA.png}
\end{center}
\ \\
\begin{center}
\includegraphics[width=0.5\textwidth]{littlesquaresB.png}
\end{center}



\newpage
\newpage
\section{Bibliography}


\begin{enumerate}
\item Serre, Jean-Pierre. "Homologie singulière des espaces fibrés. Applications." Annals of Mathematics 54, no. 3 (1951): 425-505.
\end{enumerate}



\newpage 
\ \\
\ \\
\ \\
\ \\
\ \\               
\ \\
%LEAN: 
\begin{center}
\begin{tcolorbox}[width=5in,colback={white},title={\begin{center}\texttt{About the Author} \addtocounter{lcounter}{1}  \end{center}},colbacktitle=Yellow,coltitle=black]
Dean Young is a master's student at New York University, where he studies mathematics. \\
\begin{center}
\includegraphics[width=7.5cm,height=5cm]{about.jpg}
\end{center}
\end{tcolorbox}
\end{center}

\begin{center}
\begin{tcolorbox}[width=5in,colback={white},title={\begin{center}\texttt{About the Author} \addtocounter{lcounter}{1}  \end{center}},colbacktitle=Yellow,coltitle=black]
Jiazhen Xia is a graduate student at Zhejiang University, where he studies computer science. \\
\begin{center}
\includegraphics[width=7.5cm]{about2.jpg}
\end{center}
\end{tcolorbox}
\end{center}
\newpage
\ \\
\thispagestyle{empty}
\pagecolor{Yellow}



\end{document}







































